\documentclass[12pt]{article}

\usepackage[13643]{easymcm}  % Team control number
\problem{A}  % Problem number

\title{Paper Name}  % Title

\begin{document}

\begin{abstract}
    Here is the abstract of your paper.

    Firstly, that is ...

    Secondly, that is ...

    Finally, that is ...

    \vspace{5pt}
    \textbf{Keywords}: MATLAB, mathematics, LaTeX.

\end{abstract}

\maketitle
\tableofcontents





\section{Introduction}

\subsection{Problem Background}

	Here is the problem background ...
	
	
	
	Two major problems are discussed in this paper, which are:
	\begin{itemize}
	    \item Doing the first thing.
	    \item Doing the second thing.
	\end{itemize}

\subsection{Problem Restatement}

	Blablabla
	
	
	


\section{Literature Review}

	A literature \autocite{1} say something about this problem ...  This literature \autocite{2} provides information as well ...





\section{Model for Problem 1}

	\subsection{Assumptions and Justifications}
	
		Blablabla
	
	
	
	\subsection{Variables}
	
		The primary notations used in this paper are listed in Table \ref{tb:notation}.
		
		\begin{table}[!htbp]
		\begin{center}
		\caption{Notations}
		\begin{tabular}{cl}
			\toprule
			\multicolumn{1}{m{3cm}}{\centering Symbol}
			&\multicolumn{1}{m{8cm}}{\centering Definition}\\
			\midrule
			$A$&the first one\\
			$b$&the second one\\
			$\alpha$ &the last one\\
			\bottomrule
		\end{tabular}\label{tb:notation}
		\end{center}
		\end{table}
	
	
	
	\subsection{Model Design}
	
		The detail can be described by equation \eqref{eq:heat}:
		\begin{equation}\label{eq:heat}
		\frac{\partial u}{\partial t} - a^2 \left( \frac{\partial^2 u}{\partial x^2} + \frac{\partial^2 u}{\partial y^2} + 	\frac{\partial^2 u}{\partial z^2} \right) = f(x, y, z, t)
		\end{equation}
	
	
	
	\subsection{Results}
	
		Blablabla
	
	
	
	\subsection{Sensitivity Analysis}
			
		Figure \ref{fig:subfigures} gives an example of subfigures. Figure \ref{subfig:left} is on the left, and Figure \ref{subfig:right} is on the right.
		
		\begin{figure}[htbp]
			\centering
			\begin{subfigure}[b]{.4\textwidth}
				% \includegraphics[width=\textwidth]{water.png}
				\caption{Image on the left}\label{subfig:left}
			\end{subfigure}
			\begin{subfigure}[b]{.4\textwidth}
				% \includegraphics[width=\textwidth]{water.png}
				\caption{Image on the right}\label{subfig:right}
			\end{subfigure}
			\caption{Two images}\label{fig:subfigures}
		\end{figure}
	
	
	
	\subsection{Strengths and Weaknesses}

		\subsubsection{Strengths}
		
			\begin{itemize}
			    \item First one...
			    \item Second one ...
			\end{itemize}
		
		
		
		\subsubsection{Weaknesses}
		
			\begin{itemize}
			    \item Only one ...
			 \end{itemize}





\section{Model 2}

	\subsection{Assumptions and Justifications}
	
		Blablabla
	
	
	
	\subsection{Variables}
	
		Blablabla
	
	
	
	\subsection{Model Design}
	
		Blablablablablablabla
	
	
	
	\subsection{Results}
	
		The results are shown in Figure \ref{fig:result}, where $t$ denotes the time in seconds, and $c$ refers to the concentration of water in the boiler.
		
		\begin{figure}[htbp]
			\centering
			% \includegraphics[width=.8\textwidth]{water.png}
			\caption{The result of Model 2}\label{fig:result}
		\end{figure}
		
		\begin{figure}
			\centering
			\begin{tikzpicture}
				\begin{axis}[
					title=The Square Function,
					xlabel={$x$},
					ylabel={$\sin x$}
					]
					\addplot [blue, domain=-6:6, samples=201] {x^2};
					% \addplot [blue] table {./data/TryData.dat};
				\end{axis}
			\end{tikzpicture}
		\end{figure}
	
	
	\subsection{Strengths and Weaknesses}
	
		The instance of long and wide tables are shown in Table \ref{tb:longtable}.
		
		\begin{longtable}{ p{4em} p{14em} p{14em} }
			\caption{Basic Information about Three Main Continents (scratched from Wikipedia)}
			\label{tb:longtable} \\
			\toprule
			Continent & Description & Information \\
			\midrule
			Africa & Africa Continent is surrounded by the Mediterranean Sea to the
			north, the Isthmus of Suez and the Red Sea to the northeast, the Indian
			Ocean to the southeast and the Atlantic Ocean to the west. &
			At about 30.3 million km$^2$ including adjacent islands, it covers 6\%
			of Earth's total surface area and 20\% of its land area. With 1.3
			billion people as of 2018, it accounts for about 16\% of the world's
			human population. \\
			\midrule
			Asia & Asia is Earth's largest and most populous continent which
			located primarily in the Eastern and Northern Hemispheres.
			It shares the continental landmass of Eurasia with the continent
			of Europe and the continental landmass of Afro-Eurasia with both
			Europe and Africa. &
			Asia covers an area of 44,579,000 square kilometres, about 30\%
			of Earth's total land area and 8.7\% of the Earth's total surface
			area. Its 4.5 billion people (as of June 2019) constitute roughly
			60\% of the world's population. \\
			\midrule
			Europe & Europe is a continent located entirely in the Northern
			Hemisphere and mostly in the Eastern Hemisphere. It comprises the
			westernmost part of Eurasia and is bordered by the Arctic Ocean to
			the north, the Atlantic Ocean to the west, the Mediterranean Sea to
			the south, and Asia to the east. &
			Europe covers about 10,180,000 km$^2$, or 2\% of the Earth's surface
			(6.8\% of land area), making it the second smallest
			continent. Europe had a total population of about 741 million (about
			11\% of the world population) as of 2018. \\
			\bottomrule
		\end{longtable}





\section{Conclusions}

	Blablablablabla!





\begin{letter}{Memorandum}

	\begin{flushleft}  % 左对齐环境,无首行缩进
		\textbf{To:} Heishan Yan\\
		\textbf{From:} Team 1234567\\
		\textbf{Date:} October 1st, 2019\\
		\textbf{Subject:} A better choice than MS Word: \LaTeX
	\end{flushleft}
	
	In the memo, we want to introduce you an alternate typesetting program to the prevailing MS Word: \textbf{\LaTeX}. In fact, the history of \LaTeX\ is even longer than that of MS Word. In 1970s, the famous computer scientist Donald Knuth first came out with a typesetting program, which named \TeX\ \ldots
	
	Firstly, \ldots
	
	Secondly, \ldots
	
	Lastly, \ldots
	
	According to all those mentioned above, it is really worth to have a try on \LaTeX!
	
\end{letter}





\newrefcontext[sorting=nyt]
\printbibliography

\begin{subappendices}


	\section{Appendix A: Further on \LaTeX}
	
		To clarify the importance of using \LaTeX\ in MCM or ICM, several points need to be covered, which are \ldots
		
		To be more specific, \ldots
		
		All in all, \ldots
		
		Anyway, nobody \textbf{really} needs such appendix \ldots
	
	
	
	
	
	
	\section{Appendix B: Program Codes}
	
		Here are the program codes we used in our research.
		
		\begin{lstlisting}[language=Python, name={test.py}]
		# Python code example
		for i in range(10):
		    print('Hello, world!')
		\end{lstlisting}
		
		\begin{lstlisting}[language=MATLAB, name={test.m}]
		% MATLAB code example
		for i = 1:10
		    disp("hello, world!");
		end
		\end{lstlisting}
		
		\begin{lstlisting}[language=C++, name={test.cpp}]
		// C++ code example
		#include <iostream>
		using namespace std;
		
		int main() {
		    for (int i = 0; i < 10; i++)
		        cout << 
		        "hello, world" << endl;
		    return 0;
		}
		\end{lstlisting}
	

\end{subappendices}

\end{document}
